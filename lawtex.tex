% !TeX spellcheck = de_DE
\documentclass[a4paper]{article}
\usepackage[ngerman]{babel}
\usepackage[utf8]{inputenc}
\usepackage[T1]{fontenc}
\usepackage{ebgaramond}
\usepackage{newtxmath}
\usepackage{fancyvrb}
\usepackage{fvextra}
\usepackage{enumitem}
\usepackage[hidelinks]{hyperref}
\usepackage[
backend=biber,
style=rabelsz,
maxbibnames=5,
maxcitenames=5,
backref=true
]{biblatex}

\renewcommand\thesection{\Roman{section}.}
\renewcommand\thesubsection{\arabic{subsection}.}
\renewcommand\thesubsubsection{\alph{subsubsection})}

\let\oldfootnote\footnote
\renewcommand\footnote[1]{%
	\oldfootnote{\hspace{0.5em}#1}}

\bibliography{bibliography}

\title{RabelsZ Zitierweise in \LaTeX}
\author{Corinna Coupette}
\date{}


\begin{document}

\maketitle

\section{Allgemeines}

Work in Progress.

Aktuell unterstützt werden folgende Eintragstypen (mit den unten als unterstützt aufgeführten Feldern):
\begin{itemize}[label=--,itemsep=0em]
	\item \texttt{article}
	\item \texttt{book}
	\item \texttt{incollection}
	\item \texttt{inproceedings}
	\item \texttt{misc}
\end{itemize}
Das Verhalten für andere Eintragstypen und Felder kann unter Umständen vom gewünschten Ergebnis abweichen.

Biblatex-Konfiguration dieses Dokuments:

\begin{Verbatim}
	\usepackage[
	backend=biber,
	style=rabelsz,
	maxbibnames=5,
	maxcitenames=5,
	backref=true
	]{biblatex}
\end{Verbatim}

Für den Fußnotensatz ist \texttt{style=rabelsz} entscheidend, wobei vorausgesetzt wird, dass folgende drei Dateien vom Compiler gefunden werden können (z.B., weil sie im selben Ordner liegen):
\begin{itemize}[label=--,itemsep=0em]
	\item \texttt{rabels.bbx}: Bibliographiesatz
	\item \texttt{rabels.cbx}: Zitatsatz, aufbauend auf Bibliographiesatz
	\item \texttt{rabels.lbx}: Lokalisierung 
\end{itemize}

\texttt{backref=true} zeigt im Literaturverzeichnis an, auf welchen Seiten ein Beitrag zitiert wurde. 
Das kann unter anderem bei Monographien sinnvoll sein.

Die Stylesheets basieren auf den Dokumenten im Ordner \texttt{base}, insbesondere auf  \texttt{verbose-note.cbx}.
Für bessere Hackability habe ich den Inhalt dieses Files in \texttt{rabelsz.cbx} kopiert und angepasst, anstatt den \texttt{verbose-note}-Style einzubinden.
\texttt{rabels.bbx} baut nun nur noch auf dem in jeder modernen \LaTeX\ Distribution mitgelieferten \texttt{standard.bbx} auf; \texttt{rabels.lbx} modifiziert punktuell das ebenfalls mitgelieferte \texttt{german.lbx}.

Um das Verhalten von Biblatex zu verstehen und zu wissen, was man in den Styles wo und wie anpassen muss, hilft manchmal ein Blick in die \texttt{biblatex.*}-Files im Ordner \texttt{base}, 
häufiger aber ein Blick in \texttt{standard.bbx} oder in das Biblatex-Manual im Ordner \texttt{docs}.


\section{Umsetzung der Zitierhinweise}

Quelle: \hyperref{https://www.mpipriv.de/862378/03\_Stylesheet-fuer-Autor\_innen.pdf}{}{}{Website des MPIPRIV} (zuletzt aufgerufen am \today)

Dieses Dokument illustriert die Umsetzung der \emph{Zitierhinweise}, nicht aber die Umsetzung der allgemeinen Satzhinweise. 
Die Reihenfolge der Darstellung entspricht der Reihenfolge in den Originalhinweisen. 

Die in den Beispieleinträgen verwendeten Felder werden garantiert unterstützt. 
Das Verhalten bei Verwendung anderer Felder ist nicht vollständig getestet.
Wenn der \texttt{shorttitle} fehlt, wird im Folgezitat der \texttt{title} an seiner Statt aufgeführt.

\subsection{Schrifttum}

\subsubsection{Monographien, Lehrbücher etc.}
	
\begin{Verbatim}
	@book{book:example,
		title={Book Title in Long Form},
		shorttitle={Short Title},
		author={Lastname1, Firstname1 and Lastname2, Firstname2},
		year={2000},
		edition={2},
		pagetotal={150},
		publisher={Publisher}
	}
\end{Verbatim}
	
Erstzitat\footcite[10]{book:example}
Folgezitat\footcite[20]{book:example}
	
\subsubsection{Beiträge in Sammelbänden, Festschriften etc.}
	
\begin{Verbatim}
	@incollection{incollection:example, 
		author={Lastname, Firstname}, 
		title={Long Title of the Collection Contribution}, 
		shorttitle={Short Coll Contrib Title},
		booktitle={The Title of the Collection}, 
		publisher={Publisher}, 
		editor={Editor1Last, Editor1First and Editor2Last, Editor2First}, 
		edition={2},
		year={2020}, 
		pages={1--50}
	}
\end{Verbatim}
	
Erstzitat\footcite[10]{incollection:example}
Folgezitat\footcite[20]{incollection:example}

\begin{Verbatim}
	@inproceedings{inproceedings:example,
		title={Long Title of the Proceedings Contribution},
		shorttitle={Short Proc Contrib Title},
		author={SomeAuthorLast, SomeAuthorFirst},
		booktitle={Title of the Proceedings},
		pages={1--50},
		year={2015}
	}
\end{Verbatim}

Erstzitat\footcite[10]{inproceedings:example}
Folgezitat\footcite[20]{inproceedings:example}

\subsubsection{Kommentare, Handbücher, Enzyklopädien etc.}

Erstzitat
Folgezitat

\subsubsection{Zeitschriften, Archivzeitschriften und Jahrbücher mit Bandzahl}

\begin{Verbatim}
	@article{article:example,
		title={Long Title of the Article},
		shorttitle={Short Article Title},
		author={Author1Last, Author1First and Author2Last, Author2First},
		journal={Name of the Journal},
		year={2010},
		volume={10},
		number={2},
		pages={1--50},
		publisher={Publisher}
	}
\end{Verbatim}

Erstzitat\footcite[10]{article:example}
Folgezitat\footcite[20]{article:example}

	
\subsection{Internetquellen, Zeitungen, Materialien etc.}

\begin{Verbatim}
	@misc{misc:example,
		title={Title of Misc Publication},
		author={Lastname, Firstname},
		year={2005},
		howpublished={Publication Venue},
		doi={org/identifier/reference}
	}	
\end{Verbatim}

Erstzitat\footcite{misc:example}
Folgezitat\footcite{misc:example}
	
\subsection{Gerichtsentscheidungen}
	
Erstzitat
Folgezitat

\subsection{Gesetzliche Bestimmungen etc.}
	
Erstzitat
Folgezitat


\section{Sonstiges}

\subsection{Fußnotenvarianten}

Fußnote mit Erläuterung, die \texttt{footcites} benutzt:\footcites[So etwa][10]{book:example}[][20]{inproceedings:example}[anderer Ansicht][50]{article:example}

\begin{Verbatim}[breaklines=true]
	\footcites[So etwa][10]{book:example}[][20]{inproceedings:example}[anderer Ansicht][50]{article:example}
\end{Verbatim}

Fußnote mit Erläuterung, die \texttt{footnote} und \texttt{cite} benutzt (tendenziell flexibler):\footnote{So etwa \cite[10]{book:example} und \cite[20]{inproceedings:example}; anderer Ansicht \cite[50]{article:example}.
}

\begin{Verbatim}[breaklines=true]
	\footnote{So etwa \cite[10]{book:example} und \cite[20]{inproceedings:example}; anderer Ansicht \cite[50]{article:example}.
	}
\end{Verbatim}

\subsection{Styling}

Anpassung der Nummerierung (funktioniert ähnlich für alle Dokumentebenen):
\begin{Verbatim}
	\renewcommand\thesection{\Roman{section}.}
	\renewcommand\thesubsection{\arabic{subsection}.}
	\renewcommand\thesubsubsection{\alph{subsubsection})}
\end{Verbatim}

Einrückung der Fußnoten:
\begin{Verbatim}
	\let\oldfootnote\footnote
	\renewcommand\footnote[1]{%
		\oldfootnote{\hspace{0.5em}#1}}
\end{Verbatim}

\section{Beispiele aus der Praxis}

Der Beitrag, mit dem dieses Projekt begann:\\

\noindent\cite{coupette:2022:rechtsstrukturvergleichung}

\newpage

\printbibliography

\end{document}